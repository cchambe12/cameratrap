\documentclass[11pt,a4paper]{article}\usepackage[]{graphicx}\usepackage[]{color}
%% maxwidth is the original width if it is less than linewidth
%% otherwise use linewidth (to make sure the graphics do not exceed the margin)
\makeatletter
\def\maxwidth{ %
  \ifdim\Gin@nat@width>\linewidth
    \linewidth
  \else
    \Gin@nat@width
  \fi
}
\makeatother

\definecolor{fgcolor}{rgb}{0.345, 0.345, 0.345}
\newcommand{\hlnum}[1]{\textcolor[rgb]{0.686,0.059,0.569}{#1}}%
\newcommand{\hlstr}[1]{\textcolor[rgb]{0.192,0.494,0.8}{#1}}%
\newcommand{\hlcom}[1]{\textcolor[rgb]{0.678,0.584,0.686}{\textit{#1}}}%
\newcommand{\hlopt}[1]{\textcolor[rgb]{0,0,0}{#1}}%
\newcommand{\hlstd}[1]{\textcolor[rgb]{0.345,0.345,0.345}{#1}}%
\newcommand{\hlkwa}[1]{\textcolor[rgb]{0.161,0.373,0.58}{\textbf{#1}}}%
\newcommand{\hlkwb}[1]{\textcolor[rgb]{0.69,0.353,0.396}{#1}}%
\newcommand{\hlkwc}[1]{\textcolor[rgb]{0.333,0.667,0.333}{#1}}%
\newcommand{\hlkwd}[1]{\textcolor[rgb]{0.737,0.353,0.396}{\textbf{#1}}}%
\let\hlipl\hlkwb

\usepackage{framed}
\makeatletter
\newenvironment{kframe}{%
 \def\at@end@of@kframe{}%
 \ifinner\ifhmode%
  \def\at@end@of@kframe{\end{minipage}}%
  \begin{minipage}{\columnwidth}%
 \fi\fi%
 \def\FrameCommand##1{\hskip\@totalleftmargin \hskip-\fboxsep
 \colorbox{shadecolor}{##1}\hskip-\fboxsep
     % There is no \\@totalrightmargin, so:
     \hskip-\linewidth \hskip-\@totalleftmargin \hskip\columnwidth}%
 \MakeFramed {\advance\hsize-\width
   \@totalleftmargin\z@ \linewidth\hsize
   \@setminipage}}%
 {\par\unskip\endMakeFramed%
 \at@end@of@kframe}
\makeatother

\definecolor{shadecolor}{rgb}{.97, .97, .97}
\definecolor{messagecolor}{rgb}{0, 0, 0}
\definecolor{warningcolor}{rgb}{1, 0, 1}
\definecolor{errorcolor}{rgb}{1, 0, 0}
\newenvironment{knitrout}{}{} % an empty environment to be redefined in TeX

\usepackage{alltt}
\usepackage[top=1.00in, bottom=1.0in, left=1.1in, right=1.1in]{geometry}
\usepackage{graphicx}
%\usepackage[T1]{fontenc}

%\signature{}
\IfFileExists{upquote.sty}{\usepackage{upquote}}{}
\begin{document}
\section*{Overall Goal:}
\begin{enumerate}
\item Provide a standardized method for identifying camera trap photos as accurately and efficiently as possible.
  \begin{enumerate}
  \item \emph{Ideally} we create a tool that:
    \begin{enumerate}
    \item Allows users to easily upload photos to an open source repository that is also accessible after photos have been identified
    \item When photos are uploaded to this tool/interface/whatever, they are given a standardized naming (e.g., [camera.name]\_[date.range]\_[latitudexlongitude]\_[image.num].jpg) and then the interface will sort the photos on this open source repository accordingly (i.e., by camera, date, location, and time)
    \item To create a tool that is accessible to users that have little to no experience coding
    \item To make an end product that is:
      \begin{enumerate}
      \item An excel sheet with the columns: image\_name, camera\_name, date, time, latitude, longitude, temperature, Genus, species. Genus and species could be 'blank' or 'vehicle'. 
      \item Within in excel sheet, have hyperlinks to images so user can easily download photos or verify model. \emph{Ideally} hyperlinks allow users to look at the images right before and after that image as well
      \item Bar charts [see attached example in email]
      \end{enumerate}
    \end{enumerate}
  \end{enumerate}
\end{enumerate}

\section*{Next steps:}
\begin{enumerate}
\item Look into Fraser Shillings new tool: https://wildlifeobserver.net/
\item Verify Mikey Tabak's tool: https://github.com/mikeyEcology/MLWIC
\item Do some initial investigating into alternatives options. Check Cat's notes in attached `MachineLearning\_notes.pdf'.
\item Find a good place to store images --- ideally not on TNC's box. 
\end{enumerate}

\section*{Possible Directions:}
\begin{enumerate}
\item Use MLWIC repo to make a first pass at images to weed out humans, blanks and cars when the model is $>$98.5\% confident. Then...
  \begin{enumerate}
  \item Use another tool to identify the rest
  \item Or train a new machine learning tool using MLWIC package that is more accurate for the Northeast (with the potential to expand)
  \end{enumerate}
\item Find or create an entirely new tool or interface
\end{enumerate}

\section*{Some Things to Remember:}
\begin{enumerate}
\item What do we do about multiple animals in one shot? If more than one deer can the model count? Or if there are two species. 
\item Can we manage to build a tool that can differentiate between a dog and a coyote or fox?
\item Can we make a model that is expandable? That we can tweak after this program is over at the end of the summer?
\item Is there a way to make this useable or accessible without a cluster? Maybe a website or an online tool?
\end{enumerate}



\end{document}
