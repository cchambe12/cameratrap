\documentclass[11pt,a4paper]{article}\usepackage[]{graphicx}\usepackage[]{color}
%% maxwidth is the original width if it is less than linewidth
%% otherwise use linewidth (to make sure the graphics do not exceed the margin)
\makeatletter
\def\maxwidth{ %
  \ifdim\Gin@nat@width>\linewidth
    \linewidth
  \else
    \Gin@nat@width
  \fi
}
\makeatother

\definecolor{fgcolor}{rgb}{0.345, 0.345, 0.345}
\newcommand{\hlnum}[1]{\textcolor[rgb]{0.686,0.059,0.569}{#1}}%
\newcommand{\hlstr}[1]{\textcolor[rgb]{0.192,0.494,0.8}{#1}}%
\newcommand{\hlcom}[1]{\textcolor[rgb]{0.678,0.584,0.686}{\textit{#1}}}%
\newcommand{\hlopt}[1]{\textcolor[rgb]{0,0,0}{#1}}%
\newcommand{\hlstd}[1]{\textcolor[rgb]{0.345,0.345,0.345}{#1}}%
\newcommand{\hlkwa}[1]{\textcolor[rgb]{0.161,0.373,0.58}{\textbf{#1}}}%
\newcommand{\hlkwb}[1]{\textcolor[rgb]{0.69,0.353,0.396}{#1}}%
\newcommand{\hlkwc}[1]{\textcolor[rgb]{0.333,0.667,0.333}{#1}}%
\newcommand{\hlkwd}[1]{\textcolor[rgb]{0.737,0.353,0.396}{\textbf{#1}}}%
\let\hlipl\hlkwb

\usepackage{framed}
\makeatletter
\newenvironment{kframe}{%
 \def\at@end@of@kframe{}%
 \ifinner\ifhmode%
  \def\at@end@of@kframe{\end{minipage}}%
  \begin{minipage}{\columnwidth}%
 \fi\fi%
 \def\FrameCommand##1{\hskip\@totalleftmargin \hskip-\fboxsep
 \colorbox{shadecolor}{##1}\hskip-\fboxsep
     % There is no \\@totalrightmargin, so:
     \hskip-\linewidth \hskip-\@totalleftmargin \hskip\columnwidth}%
 \MakeFramed {\advance\hsize-\width
   \@totalleftmargin\z@ \linewidth\hsize
   \@setminipage}}%
 {\par\unskip\endMakeFramed%
 \at@end@of@kframe}
\makeatother

\definecolor{shadecolor}{rgb}{.97, .97, .97}
\definecolor{messagecolor}{rgb}{0, 0, 0}
\definecolor{warningcolor}{rgb}{1, 0, 1}
\definecolor{errorcolor}{rgb}{1, 0, 0}
\newenvironment{knitrout}{}{} % an empty environment to be redefined in TeX

\usepackage{alltt}
\usepackage[top=1.00in, bottom=1.0in, left=1.1in, right=1.1in]{geometry}
\usepackage{graphicx}

%\signature{}
\IfFileExists{upquote.sty}{\usepackage{upquote}}{}
\begin{document}
\section*{Camera Trap Analysis Methods:}
\textbf{Places to start:}
\begin{enumerate}
\item Laura mentioned a handful of websites to investigate
  \begin{enumerate}
  \item https://wildlifeobserver.net/ for storing images
    \begin{enumerate}
    \item Caveat: can only upload one image at a time... yikes
    \end{enumerate}
  \item https://wildlifeinsights.org/ 
    \begin{enumerate}
    \item Not all on one platform. Use either eMammal or Wild.ID. 
    \item Started with Wild.ID since it's through CI's github, download onto desktop. Does not seem compatible with laptop. Can't select `Create Project'
    \item Need to pay for eMammal - scratch that plan
    \end{enumerate}
  \item http://drivendata.co/blog/camera-trap-wildlife-winners/ 
    \begin{enumerate}
    \item Caveat: seems to be in Python and only focuses on African fauna. 
    \end{enumerate}
  \end{enumerate}
  
\item Here are a handful of other potential resources:
  \begin{enumerate}
  \item https://github.com/mikeyEcology/MLWIC
    \begin{enumerate}
    \item Code from recent set of papers (2017 and 2018) looking at different machine learning tools
    \item Seems to be in R! and based on North American species
    \end{enumerate}
\end{enumerate}
\end{enumerate}

\noindent \textbf{Potential Approaches:}
\begin{enumerate}
\item Wild.ID, Own Code, Zooniverse
  \begin{enumerate}
  \item Start with Wild.ID to use for training. Then transition to MLWIC method to set up machine learning training specific to North America. Then use this new machine learning to remove blanks and people and group data into birds and mammals. From there, can set up citizen science project.
  \end{enumerate}
\item Zooniverse, Training Own Code
  \begin{enumerate}
  \item Start with zooniverse to help train the machine learning, then make own code to sort through photos. Caveat - inappropriate photos
  \end{enumerate}
\end{enumerate}


\end{document}
